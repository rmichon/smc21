\documentclass[fontsize=12pt]{scrartcl} % A4 paper and 11pt font size
\usepackage[a4paper,margin=2cm,bottom=3cm]{geometry}
\usepackage[utf8]{inputenc} % for french
\usepackage[T1]{fontenc} % Use 8-bit encoding that has 256 glyphs
\usepackage[francais]{babel} % English language/hyphenation
\usepackage{amsmath,amsfonts,amsthm} % Math packages
\usepackage{amssymb}
\usepackage{url}

\usepackage{float}
\usepackage{graphicx}

\usepackage{listings}
\usepackage{courier}

% parameters for listings
\usepackage{color}
\definecolor{mygrey}{rgb}{0.96,0.96,0.96}
\lstset{
  tabsize=4,
  basicstyle=\ttfamily,
  backgroundcolor=\color{mygrey},
  captionpos=b,
  breaklines=true
}

%\fontsize{16pt}

\usepackage{lipsum} % Used for inserting dummy 'Lorem ipsum' text into the template
\usepackage{hyperref}

\usepackage{sectsty} % Allows customizing section commands
\allsectionsfont{\centering \normalfont\scshape} % Make all sections centered, the default font and small caps

\usepackage{eurosym}

\usepackage{fancyhdr} % Custom headers and footers
\pagestyle{fancyplain} % Makes all pages in the document conform to the custom headers and footers
\fancyhead{} % No page header - if you want one, create it in the same way as the footers below
\fancyfoot[L]{} % Empty left footer
\fancyfoot[R]{} % Empty center footer
\fancyfoot[C]{\thepage} % Page numbering for right footer
\renewcommand{\headrulewidth}{0pt} % Remove header underlines
\renewcommand{\footrulewidth}{0pt} % Remove footer underlines
\setlength{\headheight}{13.6pt} % Customize the height of the header

\numberwithin{equation}{section} % Number equations within sections (i.e. 1.1, 1.2, 2.1, 2.2 instead of 1, 2, 3, 4)
\numberwithin{figure}{section} % Number figures within sections (i.e. 1.1, 1.2, 2.1, 2.2 instead of 1, 2, 3, 4)
\numberwithin{table}{section} % Number tables within sections (i.e. 1.1, 1.2, 2.1, 2.2 instead of 1, 2, 3, 4)

\setlength\parindent{2em} % Removes all indentation from paragraphs - comment this line for an assignment with lots of text
%\setlength{\parskip}{1em}

%\renewcommand{\baselinestretch}{1.5}

%----------------------------------------------------------------------------------------
%	TITLE SECTION
%----------------------------------------------------------------------------------------

\newcommand{\horrule}[1]{\rule{\linewidth}{#1}} % Create horizontal rule command with 1 argument of height

\title{	
\normalfont \normalsize 
\textsc{UJM (Saint-Étienne) // GRAME-CNCM (Lyon) // Gipsa-Lab (Grenoble)} \\ [25pt] % Your university, school and/or department name(s)
\horrule{0.5pt} \\[0.4cm] % Thin top horizontal rule
\huge 18\textsuperscript{ème} Conférence Sound and Music Computing (SMC-21)\\ Saint-Étienne -- Juin 2021 \\ % The assignment title
\horrule{2pt} \\[0.5cm] % Thick bottom horizontal rule
}

\date{} % Today's date or a custom date

\begin{document}

\maketitle % Print the title

\tableofcontents

\section{La conférence SMC}

\subsection{Vue d'ensemble}

Que ce soit dans nos smartphones, nos ordinateurs, nos téléviseurs, nos voitures, dans la musique que l’on écoute (qu’elle soit enregistrée ou lors de concerts), etc., les \textbf{technologies du son, de l’acoustique et de la musique} sont omniprésentes dans nos vies. L’émergence de nouvelles plateformes en lien avec ces domaines (ex. \textbf{réalité virtuelle/augmentée}, \textbf{Intelligences Artificielles (IA)}, \textbf{web apps}, etc.) les placent au centre des développements actuels poussés par les géants du secteur (ex. Facebook, Google, Apple, etc.). L’informatique musicale et la création en musique contemporaine/expérimentale ont longtemps servi d’incubateur pour ces développements et s’en sont nourris. De plus, comme le souligne Ge WANG\footnote{WANG, Ge, Artful Design: Technology in Search of the Sublime, A MusiComic Manifesto, Stanford : Stanford University Press, 2018.}, le design a joué un rôle prépondérant dans ces développements, plaçant ce domaine au centre de préoccupations Stéphanoises. 

De manière plus précise, le calcul du son et de la musique (SMC -- \textit{Sound and Music Computing}) est un domaine de recherche dont le but est d'étudier les liens entre musique, acoustique et technologie d'un point de vue interdisciplinaire. SMC peut être découplé en un ensemble de sous-domaines comprenant :

\begin{itemize}
  \item \textbf{le traitement des signaux audio/acoustiques et musicaux} qui regroupe l'ensemble des techniques de traitement du signal pour l'analyse, la synthèse et la transformation des signaux musicaux et audio ;
  \item \textbf{la compréhension et la modélisation des sons et de la musique} qui regroupent la musicologie computationnelle, la récupération automatisée d'informations musicales et l'aspect informatique des sciences cognitives appliquées à la musique ;
  \item \textbf{les interfaces pour le contrôle du son et de la musique} qui sont directement liées au domaine des Interfaces Homme/Machine (IHM) ;
  \item \textbf{l'assistance à la création musicale et sonore} qui regroupe le design sonore, les langages de programmation associés à ce domaine et la composition assistée par ordinateur.
\end{itemize}

La conférence SMC (\textit{Sound and Music Computing}) a été fondée en 2004 et est le fruit de l'union des JIMs (\textit{Journées de l'Informatique Musicale})\footnote{\url{http://jim.afim-asso.org/}} française et de leur équivalent italien : les CIMs (\textit{Colloqui di Infromatica Musicale})\footnote{\url{http://www.aimi-musica.org/}}. SMC a depuis pris une ampleur considérable pour devenir une plateforme d’échange internationale pluridisciplinaire à la croisée des arts, des sciences et des nouvelles technologies, rendez-vous incontournable des communautés de traitement du signal audio, des interfaces homme/machine pour la musique, de l'informatique musicale, des sciences cognitives appliquées à la musique, etc. SMC a lieu chaque année dans une grande ville européenne (ex. Málaga, Chypre, Helsinki, Hambourg, Dublin, Athènes, etc. pour les six dernières éditions) et \textbf{réunit des conférenciers et des artistes venant des quatre coins du monde}.

SMC s'organise autour de trois grands évènements : \textbf{une école d'été} (\textit{Summer School}), \textbf{une conférence scientifique} et \textbf{un programme artistique} riche et varié avec des \textbf{concerts} et des \textbf{installations/expositions}.

\subsection{École d'été}

D'une durée de quatre jours (avant la conférence), l'école d'été de SMC offre \textbf{une série de workshops} (en Anglais) d'une journée donnés par des grands noms de notre communauté scientifique sur des \textbf{sujets de pointe} à destination d'\textbf{étudiants internationaux au profil d'excellence} en master ou en doctorat. Pour référence, les professeurs invités de l'édition 2019\footnote{\url{http://smc2019.uma.es/summerschool.html}} de l'école d'été étaient David Cuartielles (co-fondateur d'Arduino), Jorge Calvo-Zaragoza (université McGill), Koka Nikoladze (créateurs d'instruments de musique aux millions de vues sur les réseaux sociaux) et Peter Knees (université technique de Vienne). 

\subsection{Conférence scientifique}

D'une durée de \textbf{trois jours}, la conférence scientifique fait suite à l'école d'été de SMC. \textbf{Entre 150 et 200 personnes} y participent. Quatre types de présentations y sont proposées : des \textbf{présentations orales}, des \textbf{posters}, des \textbf{démos} et des \textbf{workshops}. 

Les présentations orales et les posters font l'objet de \textbf{publications scientifiques} publiées aux termes de la conférence dans des actes (\textit{proceedings}). Les publications sont choisies et validées par un comité international d'experts composé de membres de la communauté scientifique. Les présentations orales ont une durée de quinze minutes chacune et ont lieu dans un amphithéâtre. Les posters sont présentés lors de sessions spéciales dans une grande pièce/un hall où les conférenciers peuvent déambuler. 

Les démos permettent de présenter des outils/des technologies en interagissant directement avec le public. Les sessions de démo prennent la même forme que les sessions de posters : les démonstrateurs tiennent des ``stands'' dans une grande pièce où le public peut circuler. À la différence des posters et des présentations orales, les démos sont sélectionnées directement par le comité d'organisation de la conférence.

D'une durée de deux heures, les workshops sont des ateliers/cours lors desquels des outils ou des technologies de pointe sont présentés aux participants. Le nombre de participants à un workshop est généralement limité (en général 30 personnes sauf recommendation particulière de l'enseignant). Les workshops sont sélectionnés directement par le comité d'organisation de la conférence.

\subsection{Programme artistique} 

Un programme artistique riche et varié centré autour des nouvelles technologies a lieu en parallèle de la conférence. Il regroupe des concerts de musique actuelle, expérimentale/exploratoire, contemporaine, orchestrale, etc., ainsi que des installations interactives et des expositions. Traditionnellement, au moins un concert a lieu chaque jour de la conférence dans différentes salles de la ville d'accueil. De la même manière, les installations sonores et les expositions ont lieu dans différents lieux de la ville.

\section{SMC à Saint-Étienne :\\une édition centrée sur le design}

L'édition Stéphanoise de SMC mettra l'accent sur \textbf{le design et les nouvelles technologies} afin de s'inscrire dans la tradition de la ville. Ainsi, le thème suivant a été retenu pour SMC-21 : \textbf{Computer Music and Design} (informatique musicale et design). Les dates précises de cet évènement n'ont pas encore été fixées, mais nous souhaiterions qu'il ait lieu lors de la \textbf{deuxième quinzaine de juin 2021}. Cette période coincide avec la biennale du design de Saint-Étienne, il serait idéal qu'une collaboration ait lieu entre ces deux évènements que nous pensons très complémentaires. De la même manière, nous souhaiterions que la partie scientifique de SMC-21 ainsi que l'école d'été aient lieu à la cité du design afin d'affirmer cette identité et pour donner la meilleure image possible de Saint-Étienne aux participants. 

\subsection{École d'été}
\label{subsec:ss}

L'école d'été aura lieu pendant quatre jours juste avant le début de la conférence. Elle aura pour thème : \textbf{informatique musicale et design}. La cité du design serait donc le lieu parfait pour l'accueillir.

Quatre grands noms de notre domaine seront invités pour donner un cours d'une journée sur un sujet de pointe de leur choix. Nous pensons notamment aux personnes suivantes\footnote{Cette liste n'est pas arrêtées.} :

\begin{itemize}
  \item \textbf{Ge WANG} -- professeur à l'école de design (D-School) et au Center for Computer Research in Music and Acoustics (CCRMA) de l'université Stanford, co-fondateur de la startup Smule, auteur de \textit{Artful Design: Technology in Search of the Sublime, A MusiComic Manifesto} ;
  \item \textbf{François PACHET} -- directeur du Spotify Creator Technology Research Lab, auteur de \textit{Deep Learning Techniques for Music Generation}, producteur d'\textit{Hello World}, le premier album de musique intégralement composé par une intelligence artificielle ;
  \item \textbf{Yann ORLAREY} -- directeur scientifique de GRAME-CNCM, inventeur du langage de programmation Faust ;
  \item \textbf{Roger LINN} -- fondateur de Roger Linn Design, inventeur de la boîte à rythme et du \textit{LinnStrument} récompensé par une Grammy Award.
\end{itemize} 

En parallèle de ces cours d'une journée, nous souhaiterions mettre en place \textbf{une série d'ateliers pour le grand public et les scolaires} sur différents sujets autour des technologies, du son, de l'acoustique et de la musique (ex. \textbf{synthèse sonore et design sonore, programmation informatique, création de pédales d'effets de guitare, lutherie numérique, spatialisation du son, réalité virtuelle}, etc.). Le contenu et la programmation de ces ateliers seront gérés par le département de transmission de GRAME-CNCM. 

Cet évènement était traditionnellement payant pour les participants mais nous voulons le rendre \textbf{gratuit pour cette édition stéphanoise}. De plus, un processus de sélection des étudiants sera mis en place comme les années précédentes mais \textbf{la priorité sera donnée aux étudiants stéphanois}. Deux salles (une pour les ateliers de ``haut niveau'' et une pour les ateliers ``grand public'') équipées d'un vidéoprojecteur et d'un système de diffusion sonore seront nécessaires.

\subsection{Conférence scientifique}

La conférence scientifique succédera directement à l'école d'été et aura \textbf{une durée de trois jours}. Dans l'idéal, nous souhaiterions qu'elle ait lieu à la cité du design. Encore une fois, nous pensons que ce lieu permet de mettre en valeur Saint-Étienne et ses valeurs tournées vers les nouvelles technologies et le modernisme, ce qui est aussi un de nos objectifs. 

Nous estimons que le nombre de conférenciers sera de \textbf{moins de 200 personnes}, ainsi, l'auditorium de la Platine conviendrait parfaitement aux présentations orales. Les sessions de posters et de démos pourraient prendre place dans le hall de l'auditorium et les workshops dans des salles annexes. SMC étant une conférence de type ``single track'', les sessions n'ont jamais lieu en parallèle, sauf pour les workshops qui peuvent se dérouler en même temps que les sessions de présentations orales. Un maximum de deux workshops aura lieu en même temps, il serait donc nécessaire d'avoir accès à deux salles d'une trentaine de places en plus de l'auditorium.

Deux (peut être trois) grands noms de notre domaine seront invités pour donner une présentation longue de type \textit{keynote}. Nous pensons notamment à Ge WANG, François PACHET et Roger LINN (voir la section \ref{subsec:ss}).

Les sessions de présentations orales sont généralement entrecoupées de pauses cafés, un espace devra donc être prévu à cet effet. De la même manière, l'inscription à la conférence comprend généralement les déjeuners, il faudra voir si des options existent à la cité du design pour que cela soit organisé, etc.

La conférence est payante pour les conférenciers (les personnes présentant leurs travaux). Les frais d'inscription n'ont pas encore été fixés mais il sont généralement d'environ 400\euro{} pour le plein tarif et 250\euro{} pour les étudiants. Toutefois, afin de rendre SMC-21 accessible au plus grand nombre, \textbf{nous souhaitons que la conférence soit ouverte gratuitement aux auditeurs libres}. 

\subsection{Programme artistique}

Le programme artistique de SMC-21 aura lieu en parallèle de la conférence scientifique (pendant une durée de \textbf{trois jours}). Nous prévoyons qu'au moins \textbf{deux concerts} aient lieu \textbf{chaque jour} dans différents lieux de la ville afin de mettre en valeur ses atouts culturels. 

Les concerts de musique actuelle (avec une forte dimension technologique) auront lieu \textbf{au FIL}. Les concerts de musique contemporaine pour orchestre se dérouleront au \textbf{théâtre Copeau de l'opéra}. Les autres concerts prendront place à \textbf{l'auditorium de la maison de l'université sur le campus Tréfilerie}.

Après un appel à candidature, la programmation de la moitié des concerts sera décidée par un comité de sélection international régit par le directeur artistique de la conférence (music chair) en accord avec le directeur technique local qui validera la faisabilité de chaque pièce. La programmation des autres concerts sera décidée par les organisateurs de la conférence et les différents lieux d'accueil. Un concert sera notamment dédié à l'\textbf{Ensemble Orchestral Contemporain} (EOC). 

Cette formule ``mixte'' devrait permettre d'assurer la diversité la qualité de la programmation musicale de SMC-21.

En parallèle de la programmation musicale de SMC-21, un ensemble d'installations sonores interactives investiront différents lieux phares de la ville tels que le \textbf{musée de la mine}, le \textbf{musée d'art et d'industrie}, les \textbf{campus Tréfilerie et Carnot} et potentiellement la cité du design. Tout comme pour les concerts, ces installations seront sélectionnées par un comité international d'une part et par les organisateurs de la conférence, d'autre part.

Toutes les collaborations avec les lieux mentionnés ci-dessus ont déjà été validées.

Le programme artistique de SMC-21 sera \textbf{ouvert gratuitement au public} et les artistes invités n'auront pas à payer de frais d'inscription de la conférence.

\subsection{Événements annexes}

SMC-21 est l'opportunité de faire connaître le patrimoine culturel de Saint-Étienne et de ses environs à notre communauté scientifique et artistique. 

Le banquet de SMC-21 a généralement lieu le soir du deuxième jour de la conférence. Pour cette édition Stéphanoise, nous souhaiterions qu'un groupe de musique traditionnelle local anime la soirée.

Nous souhaitons également organiser des visites touristiques de Saint-Étienne et de ses alentours pour les conférenciers.

\section{Partenaires, organisateurs et comités scientifiques et artistiques}

L'organisation de SMC-21 est le fruit d'une collaboration entre le \textbf{CIEREC}\footnote{\url{https://www.univ-st-etienne.fr/fr/cierec.html}} (\textit{Centre Interdisciplinaire d'Études et de Recherches sur l'Expression Contemporaine}) de l'\textbf{université Jean Monnet de Saint-Étienne}, le \textbf{GRAME-CNCM}\footnote{\url{http://www.grame.fr}} (\textit{Centre National de Création Musicale}) de Lyon, le \textbf{Gipsa-Lab}\footnote{\url{http://www.gipsa-lab.fr}} de l'institut polytechnique de Grenoble ainsi que d'un grand nombre d'autres partenaires :

\begin{itemize}
\item l'\textbf{opéra de Saint-Étienne} : \url{http://www.opera.saint-etienne.fr} ;
\item le \textbf{FIL -- scène de musique actuelle} : \url{http://www.le-fil.com} ;
\item le \textbf{musée de la mine} : \url{http://www.musee-mine.saint-etienne.fr/} ;
\item l'\textbf{INSA} (\textit{Institut National des Sciences Appliquées}) Lyon: \url{https://www.insa-lyon.fr} ;
\item l'\textbf{université de Lyon} : \url{https://www.universite-lyon.fr/}
\item le \textbf{CCRMA} (\textit{Center for Computer Research in Music and Acoustics}) de l'\textbf{université Stanford} : \url{https://ccrma.stanford.edu/~rmichon}
\item et nous l'espérons, la \textbf{cité du design}: \url{https://www.citedudesign.com}.
\end{itemize}

Une vue d'ensemble des membres du comité d'organisation de SMC-21 est présentée dans le tableau suivant :

\begin{table}[!htbp]
  \begin{center}
    \begin{tabular}{c | c}
      \textbf{Nom} & \textbf{Role} \\
      \hline
      \hline
      Romain MICHON & organisateur/président (Principal Chair) \\
      Laurent POTTIER & organisateur/président (Principal Chair) \\
      Yann ORLAREY & président du comité scientifique (Paper Chair) \\
      Constantin BASICA & président du comité artistique (Music Chair) \\
      James LEONARD & président de l'école d'été (Summer School Chair) \\
      Jérôme VILLENEUVE & président de l'école d'été (Summer School Chair) \\
      Philippe ROIRON & directeur technique \\
      Catinca DUMITRASCU & transmission/médiation/liens avec le public \\
    \end{tabular}
  \end{center}
\end{table}

\section{Contacts}

\begin{itemize}
\item Romain Michon, GRAME-CNCM -- \texttt{michon@grame.fr} -- 07 67 39 72 40
\item Laurent Pottier, UJM -- \texttt{laurent.pottier@univ-st-etienne.fr} % -- TODO: tel Laurent ?
\end{itemize}

\end{document}
